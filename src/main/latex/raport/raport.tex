
\documentclass[a4paper,11pt]{article}
\usepackage{latexsym}
\usepackage{graphicx}
\usepackage[polish]{babel}
\usepackage[utf8]{inputenc}
\usepackage[MeX]{polski}
\usepackage{hyperref} 
\graphicspath{{img/}}

\hypersetup{
 colorlinks   = true, %Colours links instead of ugly boxes
 urlcolor     = blue, %Colour for external hyperlinks
 linkcolor    = blue, %Colour of internal links
 citecolor   = red %Colour of citations
}

\author{Wiktor Kierzek $\quad$ Paweł Mendelski}

\title{Eksploracja danych \\ 
\large{\bf Weka -- Sprawozdanie I}} 

\begin{document} 

\maketitle 
 
\section{Wstęp}
 
Projekt polega na:
\begin{enumerate}
	\item Napisaniu własnego klasyfikatora typu \textit{Naive Bayes} w środowisku \textit{Weka}.
	\item Porównaniu otrzymanych wyników z klasyfikatorem \textit{Naive Bayes} wbudowanym w \textit{Wekę} oraz jednym dowolnie wybranym klasyfikatorem tej bilioteki.
\end{enumerate}

\section{Opis danych}

% Wykorzystane zbiory danych, krótki opis, max. pół  strony. Jeżeli dokonano ekstrakcji cech należy również to opisać i podać powód.
W eksperymencie wykorzystano 3 zbiory danych pobrane ze strony Uniwesytetu Luiven \url{http://dtai.cs.kuleuven.be/DataMiningInPractice/index.php?CONT=datasets}. 

\section{Wykorzystane miary}
Opisać miary jakie wykorzystaliście podczas eksperymentu, wraz ze wzorami i wyjaśnieniem dlaczego akurat te.
(maks. 1/2 strony)


\section{Opis eksperymentu}
Jak przebiegał cały eksperyment, max pół strony.

\section{Wyniki}
Należy podać tabelki z wynikami, mogą być wykresy. Wszystkie wyniki należy opisać. Suche dane nic nie znaczą.

\section{Podsumowanie}
Krótkie wnioski z eksperymentu

\end{document}  
